\documentclass[12pt]{article}
\textwidth=7in
\textheight=9.in
\topmargin=-0.75in
\headheight=0in
\headsep=.5in
\hoffset  -.85in

\pagestyle{empty}

\usepackage{amsmath}  %to improve the appearance of fractions

\usepackage{hyperref}
\renewcommand{\thefootnote}{\fnsymbol{footnote}}
\begin{document}

\begin{center}
    {\bf Introduction to Linux}\\
    {Course Syllabus; EE 1489 - Fall 2015}\\
    { U.S. Coast Guard Academy Department of Engineering}
\end{center}

\setlength{\unitlength}{1in}

\begin{picture}(6,.1)
    \put(0,0) {\line(1,0){6.25}}
\end{picture}

\renewcommand{\arraystretch}{2}

\vskip.1in
\noindent\textbf{Instructor:}\\
    LT Aaron P. Dahlen \\
    McAllister Hall room 211B \\
    Phone: 860-444-8318,  email: \href{Aaron.P.Dahlen@USCGA.edu}{Aaron.P.Dahlen@USCGA.edu}


\vspace*{.15in}
\noindent \textbf{Course Format:}\\
Classes will meet once a week for 75 minutes.  Expect a mixed class with lecture and lab time.

\vspace*{.15in}
\noindent \textbf{Course Description:}\\
This course is an introduction to fundamentals of Linux and Unix based systems, with an emphasis on command-line control. The overall goal of this course is to give cadets an introduction to basic concepts of Linux in general and the structure and syntax of the Debian environment in particular. Most of the learning accomplished as part of this course is in the lab and experimenting out of class. At the end of the course students will be novice Linux users who can manipulate the OS from the command-line.

This course is centered around the Raspberry Pi computer running the Raspbian operation system - a modern incarnation of Linux based on Debian. The Raspberry Pi is a popular platform with a wealth of on-line resources targeted for education and Linux newbies. The Pi has also been featured in several EE capstone projects. Each student will be issued a pi computer and accessories.



\vspace*{.15in}
\noindent\textbf{Prerequisites:}\\
1224 or permission of instructor

\vspace*{.15in}
\noindent\textbf{Textbook:}\\
``Linux Fundamentals" by Paul Cobbaut: \href{http://linux-training.be/files/books/LinuxFun.pdf}{http://linux-training.be/files/books/LinuxFun.pdf} \\
``The Linux Command Line" 2nd Internet Edition by William E. Shotts, Jr.: \href{LinuxCommand.org}{LinuxCommand.org}

\vspace*{.15in}


\vspace*{.15in}
\noindent\textbf{Course Web Page:}\\
Course material will be posted on \href{github.com/macee/tux\_15}{github.com/macee/tux\_15}. The site includes links to a complete syllabus, course schedule, homework format requirements, and selected homework/quiz/exam solutions.  All course announcements, including changes to the schedule and modifications to assignments will be announced via email.




\vspace*{.15in}
\noindent \textbf{On-line Resources:}
\begin{itemize}
    \item \href{https://www.raspberrypi.org/}{Raspberry Pi}
    \item \href{https://www.raspbian.org/}{Raspbian}
    \item \href{https://www.raspberrypi.org/magpi}{MagPi magazine}
\end{itemize}


\newpage







\vspace*{.15in}
\noindent\textbf{Policies:}

\vspace*{.15in}
\noindent\textbf{Attendance:} You are expected to attend all classes and labs. If you must miss a session due to illness, injuries, or other emergency, contact your instructor as soon as practical. If you expect to miss a session because of an extracurricular activity, contact your instructor in advance. In either case, you will want to work with a shipmate to obtain copies of class notes, assignments, etc. Also, if you expect to miss class on a date that an assignment is due, you are responsible for seeing that your work is submitted on time.

\vspace*{.15in}
\noindent\textbf{Preparedness:}  To prepare for lectures you will be given a “Pre-Underway” assignment to be submitted via D2L two hours before class begins. Each assignment consists of three to four open ended questions that are designed to ascertain basic comprehension of the material and serve as a feedback mechanism between student and instructor. Assignments are graded on a pass / fail basis. Full credit is given for answers that are lucid even if not necessary correct. The objective is to convince the instructor that you have considered the material and are ready to attend class. These assignments, along with attentiveness in class will make up your class participation grade.


\vspace*{.15in}
\noindent\textbf{Collaboration:} Collaboration on textbook based homework is allowed and encouraged. Homework assignments may be discussed with other members of the class or with anyone else whom students believe may be of assistance. However, submitted solutions must be the students own work. Students shall not collaborate on in class of take home quizzes. All resources used (excluding class text and notes), collaboration, and assistance received must be cited on submitted work. While preparing homework solutions, students are authorized to use any available text, online-documentation, or other reference materials. Copying another student’s work or course solutions (either past or present) is expressly prohibited.




\vspace*{.15in}
\noindent\textbf{Plagiarism:} This class is unique in that students are encouraged and even expected to leverage the vast amount of Raspberry Pi code that is freely available on the Internet.  However, students must be careful to avoid plagiarism when other people's code is used or modified.  At a minimum students must:
\begin{enumerate}
    \item abide by copyright law.  The “fair use” limitations are normally applicable as defined in (17 U.S.C. \textsection\ 107).  Thankfully, the vast majority of Pi code is covered under the GNU GPL or otherwise part of the public domain.

    \item provide citations for all copied code.

    \item study the code and have a general understanding of its operation.  The instructor will likely ask you how the code operates.

    \item provide the citation if there is any doubt as to the codes originality.
\end{enumerate}








\vspace*{.15in}



\newpage

\vspace*{.15in}
\noindent\textbf{Assignments and Grading:}







\vspace*{.15in}
\textbf{Quizzes:}  In-class quizzes may be assigned covering the material that was presented in the previous class period(s).  Quizzes may take the form of take home programming assignments.


\vspace*{.15in}
\textbf{Task based assignments:}
The task based exercises are self-contained learning modules with their own rubrics. All assignments must be submitted in order to pass the class.


\vspace*{.15in}
\textbf{Journal:}
In the journal you will keep a short record of your Linux activities. The journal will include such items as definitions for common commands and a record of software you installed. From time to time you will post a command line history. This journal will be posted to a github.com account.



\vspace*{.15in}
\textbf{Homework:}  Daily homework will be assigned reflecting the reading material and/or class lessons.  In addition to textbook problems, students can expect several programming assignments for the Raspberry Pi.  Homework is graded on a 10-point scale as shown in the chart below.  Homework will be docked 10\% for each day late.

    \begin{center}
        \begin{tabular}{ | l | c | }
            \hline
        %  \multicolumn{2}{|c|}{\textbf{Homework Grade Assigned}} \\
        % \hline
            \textbf{Homework Grade Assigned} & \textbf{Reflection of the homework content/correctness} \\ \hline
            10 & All problems successfully solved, comprehension clearly evident\\ \hline
            8 & Majority of problems successfully solved, comprehension present\\ \hline
            4 & Few problems successfully solved, little comprehension evident \\ \hline
            0 &  Homework not received, no comprehension evident\\
            \hline
        \end{tabular}
    \end{center}




\newpage
\vspace*{.15in}
\noindent\textbf{Course Grades:} Student grading is based on the following:

\begin{center}
     \begin{tabular}{ | l | c | }
        \hline
        \textbf{Item} & \textbf{Weight}                        \\
    \hline
        Task based assignments              & 65\%             \\
        Journal as posted to github.com     & 10\%             \\
        Timed in class exercises (quizzes)  & 10\%             \\
        Homework (graded for completion)    & 10\%             \\
        Class participation                 & 5\%              \\
    \hline
    \end{tabular}
\end{center}




\vskip.25in
\noindent\textbf{Important Dates:}
\begin{center} 
    \begin{minipage}{5in}
        \begin{flushleft}
            First day                   \dotfill 24 Aug\\
            Labor Day                   \dotfill 07 Sep (Tuesday 08 is a Monday schedule)\\
            Columbus Day                \dotfill 12 Oct\\
            Veterans' Day               \dotfill 11 Nov\\
            Thanksgiving Leave          \dotfill 25 - 29 Nov\\
            Last day                    \dotfill 9 Dec \\
            Final date to turn in work  \dotfill 15 Dec\\
        \end{flushleft}
    \end{minipage}
\end{center}




\vskip1in
\begin{center} 
    \begin{minipage}{5.5in}
        \begin{flushright}
            \begin{it} The policies and procedures outlined in this document may be changed due to extenuating circumstances or as agreed upon by instructor and students.
            \end{it}
        \end{flushright}
    \end{minipage}
\end{center}


\end{document}